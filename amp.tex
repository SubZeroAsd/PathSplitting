\section{Atomic Multi-Path}

The AMP protocol\footnote{https://lists.linuxfoundation.org/pipermail/lightning-dev/2018-February/000993.html} works as follows:

\begin{itemize}
	\item The sender builds the base preimage $\textit{BP}$, from which n 
	partial preimages will be derived. Next, the sender samples n additive shares $s_1,\ldots, s_n$, and takes the sum to compute 
	$\textit{BP} = s_1 + \ldots + s_n$.
	
	\item The sender now moves on to constructing the
n partial payments. The sender deterministically
computes the partial preimage $r_i = H(\textit{BP} ||  i)$. The payment 
condition is then $h_i = H(r_i)$. 

	\item The sender initiates the $i$-th payment by
constructing a route to the destination with 
value $v_i$ and payment hash $h_i$.
The tuple $(\textit{ID}, n, s_i)$ is included in the layered encryption 
to be opened by the receiver. {\bf Pedro: It is really important 
that here only the receiver gets the tuple $(\textit{ID}, n, s_i)$}
\end{itemize}


 In AMP, layered encryption can be only created if the sender knows the nodes in the paths from the sender to the receiver. One could imagine a little modification to overcome this in the \emph{non-source} routing setting. In particular, the sender creates an encryption $c_i := \textsf{Enc}_{\textsf{pk}_{\textit{receiver}}}(\textit{ID}, n, s_i)$. Then, the sender sends it along the payment value $v_i$ and the hash value $h_i$ to each path. \textbf{I think this is what Giulio meant as little modification.}
 
