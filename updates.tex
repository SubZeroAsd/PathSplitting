\section{Updates Since Last Meeting}

\subsection{Attack Models}
\label{sec:attackermodels}

\begin{table*}[htb]
	\centering
	\begin{tabular}{|l|c|c|c|} \hline
		&	Malicious Receiver	&	Malicious Intermediary	&	Malicious Receiver \& Intermediary \\ \hline
		Single-trapdoor & {\cmark$^*$} & \xmark & \xmark \\ \hline
		Multi-trapdoor 	& \cmark & \cmark & ? \\ \hline
	\end{tabular}
	
	{\footnotesize $^*$If we do not consider partial (early) release as an attack.}
	\label{tab:attackermodels}
	\caption{Attack models against our constructions.}
\end{table*}

The list of attacker models that we consider, and whether our constructions provide security 
against them are shown in Table \ref{tab:attackermodels}. A checkmark (\cmark) indicates that our 
construction is security in the model, and an xmark (\xmark) indicates that there exists an 
attack against our construction in the model.

\paragraph{Malicious Receiver} The only difference between our single-trapdoor and 
multi-trapdoor constructions is that in the multi-trapdoor construction we are forcing 
the receiver to wait to receive all the locks intended for him, before obtaining the 
trapdoor and starting the release. Hence, if we do not consider partial (early) release 
of the locks by the receiver as an attack, then both our construction achieve the same 
security guarantees against a malicious receiver.

\MESSAGEP{What kind of attack from the receiver actually even makes sense? After all the 
receiver just wants to receive money, and is not paying anyone.}

\paragraph{Malicious Intermediary} In case of a malicious intermediary our single-trapdoor 
construction does not provide adequate security. The reason for this is that a malicious 
intermediary who receives a lock/value pair $(\ell, \mathsf{amt})$, can split and produce 
two randomized locks $(\ell', \mathsf{amt}-\mathsf{dust})$ and $(\ell', \mathsf{dust})$ 
(for $\mathsf{dust} \ll \mathsf{amt}$), and forward the one with $\mathsf{dust}$ amount to 
the next intermediary. Afterwards, in the release phase it will learn the trapdoor and receive 
$\mathsf{amt}$ amount form the intermediary before him, even though he paid just $\mathsf{dust}$ 
amount to the intermediary after him. This attack is mitigated in our multi-trapdoor 
construction as we have the locks tied to unit amounts. Hence, an intermediary can only 
receive as many coins as it forward to the intermediary after him.

\paragraph{Malicious Receiver \& Intermediary} Previously described attack about the malicious 
intermediary in single-trapdoor construction persists here too. However, this attack is even 
more powerful now if we consider receiver/intermediary collusion. As in this case, the receiver 
can just give the trapdoor to the intermediary. 
%In multi-trapdoor construction the receiver 
%does not have the trapdoor to share with the intermediary, however he has the encrypted 
%trapdoors, and decryption key that can decrypt the ciphertexts that are part of the locks. 
%Hence, the receiver can share the encrypted trapdoors and the decryption key with the 
%intermediary, who can use the key to decrypt the ciphertexts that are part of the locks and 
%that pass through him. By doing so, the intermediary collects the key shares that he needs to 
%reconstruct the key needed to decrypt the ciphertexts that contain the trapdoors. 
%Depending on the sharing threshold used, the intermediary might get luck, and collect enough 
%shares that at the end it can reconstruct the trapdoor, and perform the above mentioned attack.

\MESSAGEP{Need to discuss the consequences of the receiver/intermediary collusion for multi-trapdoor construction.}

\subsection{Construction with Single Trapdoor}

It was suggested to look into a construction that allows an intermediary to create a mask and 
do the split as $(\ell \cdot \ell_\mathsf{mask}, \mathsf{dust})$ and $(\ell_\mathsf{mask}, 
\mathsf{amt}-\mathsf{dust})$. The hope is that such a construction will prevent the attack of 
a malicious intermediary in the single-trapdoor construction described in 
\cref{sec:attackermodels}. However, a malicious intermediary can just ignore and not use the 
mask at all, and then you are back in the scenario described above. As long as we are not using 
unit amount, or we do not tie the amounts and trapdoors somehow (such that an update to one 
affects the other), such attacks seem to be unpreventable. A malicious intermediary can create 
any form of a lock, and attack any amount to it. 

\MESSAGEP{We need to discuss this in more detail.}

\clearpage