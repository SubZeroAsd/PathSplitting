\section{Updates Since Last Meeting}

\subsection{Attacker Models}
\label{sec:attackermodels}

\begin{table*}[htb]
	\centering
	\begin{tabular}{|l|c|c|c|} \hline
		&	Malicious Receiver	&	Malicious Intermediary	&	Malicious Receiver \& Intermediary \\ \hline
		Single-trapdoor & \xmark & \xmark{$^*$} & \xmark \\ \hline
		Multi-trapdoor 	& \cmark & \cmark & \cmark \\ \hline
	\end{tabular}
	
	{\footnotesize $^*$Attack exists, but not against our currently defined security properties.}
	\caption{Attack models against our constructions.}
	\label{table:models}
\end{table*}

The list of attacker models that we consider, and whether our constructions provide security 
against them are shown in Table \ref{table:models}, where providing security means that 
a construction satisfies the properties of interest defined in \cref{sec:discussion}. 
A checkmark (\cmark) indicates that our construction satisfies the properties (i.e., is secure) 
in the model, and an xmark (\xmark) indicates that there exists an attack in the model.

\paragraph{Malicious Receiver} The only difference between our single-trapdoor and 
multi-trapdoor constructions is that in the multi-trapdoor construction we are forcing 
the receiver to wait to receive all the locks intended for him, before obtaining the 
trapdoor and starting the release. Hence, single-trapdoor construction does not fully 
achieve the atomicity property as defined in \cref{sec:discussion}, while in our definition 
of atomicity we assume that the receiver cannot start releasing without receiving all the 
locks intended for him.

\paragraph{Malicious Intermediary} In case of a malicious intermediary our single-trapdoor 
construction does not provide adequate security. The reason for this is that a malicious 
intermediary who receives a lock/value pair $(\ell, \mathsf{amt})$, can split and produce 
two randomized locks $(\ell', \mathsf{amt}-\mathsf{dust})$ and $(\ell', \mathsf{dust})$ 
(for $\mathsf{dust} \ll \mathsf{amt}$), and forward the one with $\mathsf{dust}$ amount to 
the next intermediary. Afterwards, in the release phase it will learn the trapdoor and receive 
$\mathsf{amt}$ amount from the intermediary before him, even though he paid just $\mathsf{dust}$ 
amount to the intermediary after him. This attack is mitigated in our multi-trapdoor 
construction as we have the locks tied to unit amounts. Hence, an intermediary can only 
receive as many coins as it forward to the intermediary after him.

\MESSAGEE{This described attack by default does not break any of the properties that we 
have defined in \cref{sec:discussion}. Therefore, we need to extend our definitions to 
capture this attack. One way to do this is to say that we require an intermediary only 
to be able to release its left lock(s), when all of his right locks are released. However, 
this does not take into account that some payment paths may fail, while others succeed. 
In this case it might be natural for the intermediary to receive the money only on the 
payment paths that succeeded. Do we want to model this or just force that either all 
paths succeeded or none go through?}

\paragraph{Malicious Receiver \& Intermediary} Previously described attack about the malicious 
intermediary in single-trapdoor construction persists here too. However, this attack is even 
more powerful now if we consider receiver/intermediary collusion. As in this case, the receiver 
can just give share trapdoor to the intermediary. This will break the consistency property. 
In multi-trapdoor construction the receiver does not have the trapdoors in the clear, and needs 
to wait to receive all the locks before it can reconstruct enough shares to obtain the trapdoors. 
Hence, it has nothing to share with an intermediary until it receives all the locks intended for 
him. Furthermore, as argued above, in multi-trapdoor construction one cannot further split down 
to $\mathsf{dust}$ amounts, as we assume usage of unit amounts.

\subsection{New Construction with Single Trapdoor}

It was suggested to look into a construction that allows an intermediary to create a mask and 
do the split as $(\ell \cdot \ell_\mathsf{mask}, \mathsf{dust})$ and $(\ell_\mathsf{mask}, 
\mathsf{amt}-\mathsf{dust})$, where additionally, the encryption of $\mathsf{mask}$ is given to 
the receiver.. The hope is that such a construction will prevent the attack of a malicious 
intermediary in the single-trapdoor construction described in \cref{sec:attackermodels}. 
However, absent of any form of proof, a malicious intermediary can just ignore and not use the 
mask at all, and then you are back in the scenario that attack applies. A malicious intermediary 
can create a valid looking lock, and attach any amount to it. That being said, if we include some 
form of correct-behavior proof from the intermediary, then this approach might work. The 
question is what exactly is the statement that the intermediary is trying to proof here?

\clearpage