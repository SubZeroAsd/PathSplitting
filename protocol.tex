\section{Protocol}

\TODOE{Building blocks}

\TODOE{Overview}

\subsection{Intuition}

\subsection{Cryptographic Building Blocks}

\paragraph{Bilinear Groups} Let $\BilGen$ be a $\ppt$ algorithm that on input a security 
parameter $1^\lambda$ outputs $(q, e, \GG_1, \GG_2, \GG_T, g_1, g_2, g_T) \gets \BilGen
(1^\lambda)$, where $\GG_1, \GG_2$ and $\GG_T$ are groups of prime order $q$ with bilinear 
map $e \colon \GG_1 \times \GG_2 \to \GG_T$, and generators $g_i \in \GG_i$ for $i \in \{1,2,T\}$.
We consider Type-III bilinear groups as they represent the state-of-the-art regarding the 
efficiency and security.

\paragraph{Universal Re-encryption Scheme} Universal re-encryption allows to re-encrypt without knowledge 
of the public key under which a ciphertext was computed. Therefore, one can use it to re-randomize an input 
ciphertext, such that the encrypted value remains the same, but the output ciphertext looks different than 
the input ciphertext. Golle et al. \cite{gjjs} showed how to build such a primitive using the ElGamal 
encryption scheme. This is achieved by appending to a standard ElGamal ciphertext a second ciphertext on 
the identity element. Since ElGamal is homomorphic, one can use the second ciphertext to alter the encryption 
factor in the first ciphertext. Hence, one does not need knowledge of the public key in the re-encryption 
operation.

\subsection{Single-Trapdoor Construction}

\begin{figure*}[htb]
	\begin{center}
	\framebox[0.78\textwidth]{
		\begin{minipage}[t]{0.75\textwidth}
			\underline{{$\langle \Setup_{U_0}(1^\lambda, U_n), \Setup_{U_n}(1^\lambda) \rangle$}}: 
			On input the security parameter $1^\lambda$, a user identifier $U_n$, 
			obtain the public parameters $(q, e, \GG_1, \GG_2, \GG_T, g_1, g_2, g_T) \gets \BilGen(1^\lambda)$,
			sample $(x,y) \sample \ZZ_q^2$, compute the state $s_1 \gets (g_1^{xy}, g_T^y)$, send 
			$\vec{\td} := (x)$ to $U_n$ and return $\vec{s} := (s_1)$. \\
			\underline{$\langle \Lock_{U_i}(\vec{s}, \{(U_{j_1},p_{j_1}),\ldots,(U_{j_m},p_{j_m})\}), 
			\Lock_{U_{j_1}}(), \ldots, \Lock_{U_{j_m}}() \rangle$}: On input a vector of states  
			$\vec{s} := (s_1,\ldots,s_l)$, a set of tuples $\{(U_{j_1},p_{j_1}),\ldots,(U_{j_m},p_{j_m})\}$, 
			where $U_{j_i}$ is a user identifier and $p_{j_i}$ is the corresponding partition size, 
			$\forall e$ in $[1,m]$, sample a randomness $r_e \sample \ZZ_q$, pick $p_{j_e}$ many states 
			$(s_1,\ldots,s_{p_{j_e}})$ from $\vec{s}$, $\forall a$ in $[1,p_{j_e}]$, parse state $s_a$ 
			as $(s_a^{(1)},s_a^{(2)})$, set $\ell_a \gets (s_a^{(2)})^r, s_a^\dag \gets 
			((s_a^{(1)})^r, \ell_a)$. Finally, send $\vec{s}^\dag := (s_1^\dag,\ldots,
			s_{p_{j_e}}^\dag)$ to $U_{j_e}$, and store $\vec{w} := (r_1,\ldots,r_m)$. \\
			\underline{$\Extract(\vec{s}, \vec{\td})$}: On input a vector of states $\vec{s} := (s_1,
			\ldots,s_l)$, a vector of trapdoors $\vec{\td} := (x)$, $\forall i$ in $[1,l]$, parse $s_i$ as 
			$(s_i^{(1)}, s_i^{(2)})$, check whether $e((s_i^{(1)})^{x^{-1}}, g_2) \? s_i^{(2)}$. If the 
			check passes, set $k_i = (s_i^{(1)})^{x^{-1}}$, otherwise return $\bot$. Finally, return 
			$\vec{k} := (k_1,\ldots,k_l)$. \\
			\underline{$\Release(\vec{k}, \vec{w})$}: On input a vector of keys $\vec{k} := (k_1,\ldots,k_l)$, 
			and a vector of witnesses $\vec{w} := (r_1,\ldots,r_l)$, $\forall i$ in $[1,l]$, set $k_i' \gets 
			(k_i)^{{r_i}^{-1}}$, and return $\vec{k'} := (k_1',\ldots,k_l')$. \\
			\underline{$\Verify(\vec{s}, \vec{k})$}: On input a vector of states $\vec{s} := (s_1,\ldots,
			s_l)$ and a vector of keys $\vec{k} := (k_1,\ldots,k_l)$, return 1 if $\forall i \in [1,l]$, 
			parse $s_i := (s_i^{(1)}, s_i^{(2)})$ and $e(k_i, g_2) \? s_i^{(2)}$ holds, and 
			0 otherwise.
		\end{minipage}
	}
	\end{center}
	
	\caption{Algorithms and protocols for the single trapdoor construction.}
	\label{fig:single-trapdoor}
\end{figure*}

%\MESSAGEE{Can an intermediary have unequal number of states, keys and witnesses for $\Extract$?}

\subsection{Multi-Trapdoor Construction}

\begin{figure*}[htb]
	\begin{center}
	\framebox[0.78\textwidth]{
		\begin{minipage}[t]{0.75\textwidth}
			\underline{$\langle \Setup_{U_0}(1^\lambda, v, U_n), \Setup_{U_n}(1^\lambda) \rangle$}: 
			On input the security parameter $1^\lambda$, a payment value $v$ and a user identifier $U_n$, 
			obtain the public parameters $(q, e, \GG_1, \GG_2, \GG_T, g_1, g_2, g_T) \gets \BilGen(1^\lambda)$,
			sample $(x_1,\ldots,x_v,y) \sample \ZZ_q^{v+1}$ and $k_t, k_s \sample \mathcal{K}$, secret 
			share $k_t$ as $([k_t]_1,\ldots[k_t]_v)$, generate ciphertexts $(c_t^{(1)},\ldots,c_t^{(v)}) \gets 
			(\enc(k_t, x_1),\ldots,\enc(k_t,x_v))$ and $(c_s^{(1)},\ldots,c_s^{(v)}) \gets (\enc(k_s, 
			[k_t]_1),\ldots, \enc(k_s, [k_t]_v))$, compute the states $(s_1,\ldots,s_v) \gets ((g_1^{x_1y}, 
			g_T^y, c_s^{(1)}),\ldots,(g_1^{x_vy}, g_T^y, c_s^{(v)}))$, send 
			$\vec{\td} := (k_s, (c_t^{(1)},\ldots,c_t^{(v)}))$ to $U_n$, and return 
			$\vec{s} := (s_1,\ldots,s_v)$. \\
			\underline{$\langle \Lock_{U_i}(\vec{s}, \{(U_{j_1},p_{j_1}),\ldots,(U_{j_m},p_{j_m})\}), 
			\Lock_{U_{j_1}}(), \ldots, \Lock_{U_{j_m}}() \rangle$}: On input a vector of states  
			$\vec{s} := (s_1,\ldots,s_l)$, a set of tuples $\{(U_{j_1},p_{j_1}),\ldots,(U_{j_m},p_{j_m})\}$, 
			where $U_{j_i}$ is a user identifier and $p_{j_i}$ is the corresponding partition size, 
			$\forall e$ in $[1,m]$, sample a randomness $r_e \sample \ZZ_q$, pick $p_{j_e}$ many unused 
			states $(s_1,\ldots,s_{p_{j_e}})$ from $\vec{s}$, $\forall a$ in $[1,p_{j_e}]$, parse state $s_a$ 
			as $(s_a^{(1)},s_a^{(2)},s_a^{(3)})$, set $\ell_a \gets (s_a^{(2)})^r, s_a^\dag \gets 
			((s_a^{(1)})^r, \ell_a, \rand(s_a^{(3)}))$. Finally, send $\vec{s}^\dag := (s_1^\dag,\ldots,
			s_{p_{j_e}}^\dag)$ to $U_{j_e}$, and store $\vec{w} := (r_1,\ldots,r_m)$. \\
			\underline{$\Extract(\vec{s}, \vec{\td})$}: On input a vector of states  
			$\vec{s} := (s_1,\ldots,s_l)$, a vector of trapdoors $\vec{td} := (k_s, (c_1,\ldots,c_v))$, 
			if $l \neq v$, then return $\bot$. Otherwise, $\forall i$ in $[1,l]$, parse $s_i$ as 
			$(s_i^{(1)}, s_i^{(2)}, s_i^{(3)})$, and decrypt to obtain $[k_t]_i \gets 
			\dec(k_s, s_i^{(3)})$. If any $[k_t]_i = \bot$, then return $\bot$, else reconstruct 
			$k_t$ from $([k_t]_1,\ldots,[k_t]_v)$. Then, $\forall i$ in $[1,v]$, decrypt 
			$x_i \gets \dec(k_t, c_i)$, if $x_i = \bot$, then return $\bot$, otherwise, 
			check whether $e((s_i^{(1)})^{{x_i}^{-1}}, g_2) \? s_i^{(2)}$. If the check passes, set 
			$k_i \gets (s_i^{(1)})^{{x_i}^{-1}}$, otherwise return $\bot$. Finally, return $\vec{k} := 
			(k_1,\ldots,k_l)$. \\
			\underline{$\Release(\vec{k}, \vec{w})$}: On input a vector of keys $\vec{k} := (k_1,\ldots,k_l)$, 
			and a vector of witnesses $\vec{w} := (r_1,\ldots,r_l)$, $\forall i$ in $[1,l]$, set $k_i' \gets 
			(k_i)^{{r_i}^{-1}}$, and return $\vec{k'} := (k_1',\ldots,k_l')$. \\
			\underline{$\Verify(\vec{s}, \vec{k})$}: On input a vector of states $\vec{s} := (s_1,\ldots,
			s_l)$ and a vector of keys $\vec{k} := (k_1,\ldots,k_l)$, return 1 if $\forall i \in [1,l]$, 
			parse $s_i := (s_i^{(1)}, s_i^{(2)}, s_i^{(3)})$ and $e(k_i, g_2) \? s_i^{(2)}$ holds, and 
			0 otherwise.
		\end{minipage}
	}
	\end{center}
	
	\caption{Algorithms and protocols for the multiple trapdoor construction.}
	\label{fig:multi-trapdoor}
\end{figure*}
