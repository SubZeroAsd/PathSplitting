\section{Security Model}
\label{sec:security-model}

\TODOE{Threat model, communication model, ...}

\TODOE{Write Ideal Functionality for both locks (multiple-trapdoor and single-trapdoor). Ideally, we have one ideal functionality 
where we highlight the few differences.}

In this section, we formalize the security and privacy notions of \sysname. We make use of 
the universal composability (UC) framework of Canetti \cite{canetti}.

\subsection{Attacker Model}
\label{sec:attacker-model}

We model parties as interactive Turing machines (ITMs), which communicate with a trusted 
functionality $\FF$ via secure and authenticated communication channels. We model the 
adversary $\adv$ as a $\ppt$ machine that has access to an interface $\corrupt(\cdot)$, 
which takes as input a party identifier $P$ and provides the attacker with the internal 
state of $P$. From that point onward, all subsequent incoming and outgoing communication 
of $P$ is routed through $\adv$. We consider the static corruption model, that is, the 
adversary is required to commit to the identifiers of the parties it wishes to corrupt 
ahead of time.

\subsection{Ideal Functionality}
\label{sec:ideal-func}

\paragraph{Communication Model}
Communication happens through the secure transmission functionality $\FF_{\smt}$, as 
defined in \cite{canetti}, which informs the adversary whenever a communication between 
any two parties happens, and allows the adversary to delay the delivery of the messages 
arbitrarily. However, the adversary can neither read nor change the content of the messages.

We consider a synchronous communication network, where communication proceeds in discrete 
rounds, as defined in \cite{kmtz} and denoted here as $\FF_{\syn}$. The parties are 
always aware of the current round, and if a party $P$ sends a message in round $r$, 
the recipient party $R$ receives the message in the beginning of round $r + 1$. The 
adversary can change the order of messages, but we assume that the order of messages 
between honest parties cannot be changed (which can easily be realized using message 
counters). For simplicity, we assume that computation is instantaneous.

Furthermore, we assume existence of an anonymous communication channel as defined in 
\cite{cl}, which we denote here as $\FF_{\anon}$. In the ideal functionality we use the 
interfaces send$_\smt$ and receive$_\smt$ to exchange messages through the $\FF_{\smt}$ 
functionality, and interfaces send$_\anon$ and receive$_\anon$ to exchange messages via 
$\FF_{\anon}$.

\paragraph{Our Model}
In the following we define an ideal functionality $\FF_{\sysname}$ for {\sysname}s. For a 
more modular treatment, we only model the cryptographic lock functionality.

\MESSAGEP{Do we consider unidirectional or bidirectional links?}

$\FF_{\sysname}$ works in interaction with a universe of users $\mathbb{U}$. Additionally, 
$\FF_{\sysname}$ initializes an empty list $\LL := \emptyset$, which is used to track the 
locks. The entries of $\LL$ are of the form $(\lid_i, \td_i, U_i, U_{i+1}, f, \\ \lid_{i+1})$, 
where $\lid_i$ is unique lock identifier, $\td_i$ is the trapdoor of the lock, $U_i$ and $U_{i+1}$ 
are the users connected by the lock (the left and right user, respectively), $f \in \{\Init,\Lock,
\Rel\}$ is a status flag of the lock, and $\lid_{i+1}$ is the identifier of the next lock on the 
path. We note that the trapdoors are only associated to the initial locks created for the sender, 
and in all subsequent locks on the path the trapdoor is set to $\bot$. However, since our list 
$\LL$ has a linked structure (includes the next lock identifier), we can recursively traverse the 
list of locks on the path to obtain the trapdoor associated to the path.

For ease of exposition we define several functions that extract or update information of a 
lock given its identifier as input: $\getStatus(\cdot)$ returns the status of the lock, 
$\getLeft(\cdot)$ and $\getRight(\cdot)$ return the left and right user involved with the 
lock, $\getNextLock(\cdot)$ returns the identifier of the next lock, $\setRightUser(\cdot, U)$ 
sets the right user of the lock to a user $U$, $\setRightLock(\cdot, \lid)$ sets the right lock 
of the lock to $\lid$, $\updateStatus(\cdot, f)$ sets the status of the lock to a flag $f$, 
$\generateTrapdoor(\cdot)$ creates a trapdoor associated to the lock, and $\validateTrapdoors
((\cdots), (\td_1,\ldots,\td_m))$ checks that the trapdoors are associated to the given locks 
by recursively traversing the path of each given lock.

$\FF_{\sysname}$ defines 5 interfaces. The $\Setup$ interface allows the user $U_0$ (the sender) 
to setup states and trapdoors to be used later when establishing and releasing the locks. The $\Lock$ 
interface allows a user to establish locks with its right neighbors using the previously established 
locks. The $\Extract$ interface allows the user $U_n$ (the receiver) to release the locks that it 
received using the initially created trapdoors. The $\Release$ interface allows an intermediate user 
along the path to release the locks with its left neighbors assuming its right locks are released. 
Lastly, $\mathsf{GetStatus}$ interface allows a user to check the current status of a lock. We note 
that internally the locks are assigned identifiers that are unique across all paths. Our $\FF_{\sysname}$ 
ideal functionality can be seen in \cref{fig:ideal-func}.

$\FF_{\sysname}$ makes use of $\FF_{\anon}, \FF_{\smt}$ and $\FF_{\syn}$, thus, it is defined in 
$(\FF_{\anon}, \FF_{\smt}, \FF_{\syn})$-hybrid model.

\subsection{Discussion}

We define the security and privacy notions of interest for {\sysname}s, and describe how they 
are captured by the functionality $\FF_{\sysname}$.

\paragraph{Consistency} A \sysname is consistent if no user can release its left lock without its 
right lock being released first. Our ideal functionality captures this property as the $\Release$ 
interface allows a user to release the left lock only if the corresponding right lock has already 
been released. In case the user is the receiver, the $\Extract$ interface is used for releasing, 
and it makes sure that there are no further right locks before starting the release of the provided 
locks. 

\paragraph{Atomicity} Atomicity means that every user in any path is able to release its left lock 
only if the corresponding right lock is already released. Our ideal functionality enforces this by 
keeping track of all the locks in the list $\LL$, and in the $\Release$ interface only changing the 
status of the lock to $\Rel$ if the corresponding next lock is already released (has flag $\Rel$). 
If the user is the receiver, then the $\Extract$ interface validates that no right lock exists, which 
signifies that the release can start.

\paragraph{Anonymity} Anonymity requires that each intermediate user does not learn any information 
about the set of users in a \sysname beyond its direct neighbors.
%and furthermore, it cannot distinguish if it is part of a single-path or multi-path lock. 
This is achieved by our ideal functionality since the lock identifiers are sampled at random, and 
during the locking phase a user only learns the identifiers of its left and right locks, along with 
its left and right neighbors.

\paragraph{Completeness} A \sysname is complete if the receiver can only start releasing its locks,
once it has received same number of locks as the trapdoors. More precisely, our ideal functionality 
generates $v$ trapdoors in the $\Setup$ interface, and in the $\Extract$ interface the release can 
only begin if the receiver has $v$ locks as well.

\subsection{Universal Composability}
\label{sec:uc}

We now review the notion of secure realization in the UC framework \cite{canetti}. 
Intuitively, a protocol realizes an ideal functionality if the adversary has no 
advantage in distinguishing between an ideal functionality and a real-world protocol, 
where a simulator $\sdv$ is in charge of translating the messages produced by the 
ideal functionality for a computational adversary $\adv$. We denote by $\exec_{\pi, 
\adv, \dist}$ the ensemble of the outputs of the environment $\dist$ when interacting 
with the adversary $\adv$, and the users running protocol $\pi$. Analogously, we denote  
by $\exec_{\FF, \sdv, \dist}$ the ensemble of the outputs of the environment $\dist$ 
when interacting with the simulator $\sdv$, and the ideal functionality $\FF$.

\begin{definition}[Universal Composability]
A protocol $\pi$ UC-realizes an ideal functionality $\FF$ if for any $\ppt$ adversary 
$\adv$ there exists a simulator $\sdv$, such that for any environment $\dist$, the 
ensembles $\exec_{\pi, \adv, \dist}$ and $\exec_{\FF, \sdv, \dist}$ are computationally 
indistinguishable.
\end{definition}

\begin{figure*}[!h]
	\centering
	
	\fbox{
		\begin{minipage}[t]{0.50\textwidth}
			\procedure[mode=text]{$\Setup(\sid, v, (U_0,U_n))$}{
				Upon invocation by $U_0$:  												\\
				$\forall i \in [1,v]$:													\\
				\quad $\lid_i \sample \{0, 1\}^\lambda$									\\
				\quad $\td_i \gets \generateTrapdoor(\lid_i)$							\\
				\quad insert $(\lid_i, \td_i, U_0, \bot, \Init, \bot)$ to $\LL$			\\
				send$_\anon$ $(\sid, ((\lid_1, \Init),\ldots,(\lid_v, \Init)))$ to $U_0$\\
				send$_\anon$ $(\sid, v, (\td_1,\ldots,\td_v))$ to $U_n$
			}
			\\ \\
			\procedure[mode=text]{$\Release(\sid, (\lid_1,\ldots,\lid_k))$}{
				Upon invocation by $U_i$: 												\\
				$\forall i \in [1,k]$: 													\\
				\quad if $U_i = U_n$ or $\getRight(\lid_i) \neq U_i$ or 
				$\getStatus(\lid_i) \neq \Lock$ or $\getStatus(\getNextLock(\lid_i)) 
				\neq \Rel$	then send$_\smt$ $(\sid, \lid_i, \bot)$ to $U_i$				\\
				\quad else:																\\
				\qquad $\updateStatus(\lid_i, \Rel)$										\\
				\qquad send$_\smt$ $(\sid, \lid_i, \Rel)$ to $\getLeft(\lid_i)$
			}
			\\
			\procedure[mode=text]{$\Extract(\sid, (\lid_1,\ldots,\lid_k),(\td_1,\ldots,\td_v))$}{
				Upon invocation by $U_n$:												\\
				$\forall i \in [1,k]$: if $\getRight(\lid_i) \neq U_n$ or 
				$\getStatus(\lid_i) \neq \Lock$ or 										\\
				$\getNextLock(\lid_i) \neq \bot$ then abort								\\
				if \textcolor{blue}{$k \neq v$ or} $\validateTrapdoors
				((\lid_1,\ldots,\lid_k),(\td_1,\ldots,\td_v)) = \bot$ then abort			\\
				$\forall i \in [1,k]$: 													\\
				\quad $\updateStatus(\lid_i, \Rel)$										\\
				\quad send$_\smt$ $(\sid, \lid_i, \Rel)$ to $\getLeft(\lid_i)$
			}
		\end{minipage}
		\begin{minipage}[t]{0.45\textwidth}
			\procedure[mode=text]{$\Lock(\sid, (\lid_1,\ldots,\lid_k), 
				\{(U_{j_1},p_{j_1}),\ldots,(U_{j_m},p_{j_m})\})$}{
				Upon invocation by $U_i$: 												\\
				$\forall j \in [1,k]$: if $U_i \neq U_0$ and $\getStatus(\lid_j) \neq 
				\Lock$ and $\getRight(\lid_j) \neq U_i$ the abort						\\
				\textcolor{blue}{if $\sum_{i=1}^m p_{j_i} > k$ then abort}				\\
				set $\ell = (\lid_1,\ldots,\lid_k)$										\\
				$\forall e \in [1,m]$:													\\
				\quad $\forall a \in [1,p_{j_e}]$:										\\
				\qquad $\lid' \sample \{0,1\}^{\lambda}$									\\
				\qquad sample $\lid^* \sample \ell$ and set $\ell = \ell \setminus 
				\{\lid^*\}$																\\
				\qquad if $U_i = U_0$: 													\\
				\qquad \quad $\setRightUser(\lid^*, U_{j_e})$							\\
				\qquad \quad set $\lid' = \lid^*$										\\
				\qquad else: 															\\
				\qquad \quad insert $(\lid', \bot, U_i, U_{j_e}, \Init, \bot)$ into $\LL$\\
				\qquad \quad $\setRightLock(\lid^*, \lid')$								\\
				\qquad send$_\smt$ $(\sid, \lid', \Lock)$ to $\getRight(\lid')$			\\
				\qquad receive$_\smt$ $(\sid, b)$ from $\getRight(\lid')$				\\
				\qquad if $b = \bot$ then send$_\smt$ $(\sid, \getRight(\lid'), \bot)$ 
				to $U_i$																	\\
				\qquad else:																\\
				\qquad \quad $\updateStatus(\lid', \Lock)$								\\
				\qquad \quad send$_\smt$ $(\sid, \lid', \Lock)$ to $U_i$ and $\getRight(\lid')$
			}
			\\
			\procedure[mode=text]{$\mathsf{GetStatus}(\sid, \lid)$}{
				Upon invocation by $U_i$:												\\
				%if $\getLeft(\lid) \neq U_i$ or $\getRight(\lid) \neq U_i$ then abort	\\
				send$_\smt$ $(\sid, \lid, \getStatus(\lid))$ to $U_i$
			}
		\end{minipage}
	}
	
	\caption{Ideal functionality $\FF_{\sysname}$ in $(\FF_{\anon}, \FF_{\smt}, \FF_{\syn})$-hybrid model.}
	\label{fig:ideal-func}
\end{figure*}
